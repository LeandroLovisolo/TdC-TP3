
\documentclass[a4paper, 10pt, twoside]{article}

\usepackage[top=1in, bottom=1in, left=1in, right=1in]{geometry}
\usepackage[utf8]{inputenc}
\usepackage[spanish, es-ucroman, es-noquoting]{babel}
\usepackage{setspace}
\usepackage{fancyhdr}
\usepackage{lastpage}
\usepackage{amsmath}
\usepackage{amsfonts}
\usepackage{amsthm}
\usepackage{verbatim}
\usepackage{fancyvrb}
\usepackage{graphicx}
\usepackage{float}
\usepackage{enumitem} % Provee macro \setlist
\usepackage{tabularx}
\usepackage{multirow}
\usepackage{hyperref}
\usepackage{xspace}
\usepackage[toc, page]{appendix}


%%%%%%%%%% Configuración de Fancyhdr - Inicio %%%%%%%%%%
\pagestyle{fancy}
\thispagestyle{fancy}
\lhead{Trabajo Práctico 3 · Teoría de las Comunicaciones}
\rhead{Delgado · Lovisolo · Petaccio}
\renewcommand{\footrulewidth}{0.4pt}
\cfoot{\thepage /\pageref{LastPage}}

\fancypagestyle{caratula} {
   \fancyhf{}
   \cfoot{\thepage /\pageref{LastPage}}
   \renewcommand{\headrulewidth}{0pt}
   \renewcommand{\footrulewidth}{0pt}
}
%%%%%%%%%% Configuración de Fancyhdr - Fin %%%%%%%%%%


%%%%%%%%%% Miscelánea - Inicio %%%%%%%%%%
% Evita que el documento se estire verticalmente para ocupar el espacio vacío
% en cada página.
\raggedbottom

% Separación entre párrafos.
\setlength{\parskip}{0.5em}

% Separación entre elementos de listas.
\setlist{itemsep=0.5em}

% Asigna la traducción de la palabra 'Appendices'.
\renewcommand{\appendixtocname}{Apéndices}
\renewcommand{\appendixpagename}{Apéndices}
%%%%%%%%%% Miscelánea - Fin %%%%%%%%%%


%%%%%%%%%% Insertar gráfico - Inicio %%%%%%%%%%
\newcommand{\grafico}[3]{
  \begin{figure}[H]
    \includegraphics[type=pdf,ext=.pdf,read=.pdf]{#1}
    \caption{#2}
    \label{#3}
  \end{figure}
}
%%%%%%%%%% Insertar gráfico - Fin %%%%%%%%%%


%%%%%%%%%% Palabras clave - Inicio %%%%%%%%%%
\newcommand{\established}{\texttt{ESTABLISHED}\xspace}
\newcommand{\ack}{\texttt{\#ACK}\xspace}
\newcommand{\window}{\texttt{ventana}\xspace}
%%%%%%%%%% Palabras clave - Fin %%%%%%%%%%


\begin{document}


%%%%%%%%%%%%%%%%%%%%%%%%%%%%%%%%%%%%%%%%%%%%%%%%%%%%%%%%%%%%%%%%%%%%%%%%%%%%%%%
%% Carátula                                                                  %%
%%%%%%%%%%%%%%%%%%%%%%%%%%%%%%%%%%%%%%%%%%%%%%%%%%%%%%%%%%%%%%%%%%%%%%%%%%%%%%%


\thispagestyle{caratula}

\begin{center}

\includegraphics[height=2cm]{DC.png} 
\hfill
\includegraphics[height=2cm]{UBA.jpg} 

\vspace{2cm}

Departamento de Computación,\\
Facultad de Ciencias Exactas y Naturales,\\
Universidad de Buenos Aires

\vspace{4cm}

\begin{Huge}
Trabajo Práctico 3
\end{Huge}

\vspace{0.5cm}

\begin{Large}
Teoría de las Comunicaciones
\end{Large}

\vspace{1cm}

Primer Cuatrimestre de 2014

\vspace{4cm}

\begin{tabular}{|c|c|c|}
\hline
Apellido y Nombre & LU & E-mail\\
\hline
Delgado, Alejandro N.  & 601/11 & nahueldelgado@gmail.com\\
Lovisolo, Leandro      & 645/11 & leandro@leandro.me\\
Petaccio, Lautaro José & 443/11 & lausuper@gmail.com\\
\hline
\end{tabular}

\end{center}

\newpage


%%%%%%%%%%%%%%%%%%%%%%%%%%%%%%%%%%%%%%%%%%%%%%%%%%%%%%%%%%%%%%%%%%%%%%%%%%%%%%%
%% Índice                                                                    %%
%%%%%%%%%%%%%%%%%%%%%%%%%%%%%%%%%%%%%%%%%%%%%%%%%%%%%%%%%%%%%%%%%%%%%%%%%%%%%%%


\tableofcontents

\newpage


%%%%%%%%%%%%%%%%%%%%%%%%%%%%%%%%%%%%%%%%%%%%%%%%%%%%%%%%%%%%%%%%%%%%%%%%%%%%%%%
%% Introducción                                                              %%
%%%%%%%%%%%%%%%%%%%%%%%%%%%%%%%%%%%%%%%%%%%%%%%%%%%%%%%%%%%%%%%%%%%%%%%%%%%%%%%


\section{Introducción}
\label{sec:introduccion}


En el presente trabajo se estudia el desempeño del protocolo PTC\footnote{Protocolo desarrollado por la cátedra del curso \emph{Teoría de las Comunicaciones} dictado por el Departamento de Computación de la Universidad de Buenos Aires.} ante algunos fenómenos típicos de una red de área local. El protocolo PTC es un protocolo educacional de capa de transporte que implementa la técnica de ventana deslizante. Los fenómenos estudiados son la latencia y la pérdida de paquetes.


%%%%%%%%%%%%%%%%%%%%%%%%%%%%%%%%%%%%%%%%%%%%%%%%%%%%%%%%%%%%%%%%%%%%%%%%%%%%%%%
%% Desarrollo                                                                %%
%%%%%%%%%%%%%%%%%%%%%%%%%%%%%%%%%%%%%%%%%%%%%%%%%%%%%%%%%%%%%%%%%%%%%%%%%%%%%%%


\section{Desarrollo}
\label{sec:desarrollo}

Se partió de una implementación del protocolo PTC\footnote{Esta implementación fue provista por la cátedra.} a la que se le introdujeron modificaciones para simular los fenómenos estudiados. Luego se realizaron experimentos para medir cómo impactan en el desempeño del protocolo estos fenómenos simulados.


\subsection{Comportamiento de la implementación de PTC}

En lo que sigue se describen algunos escenarios que ocurren al usar la implementación original. La motivación de esto es dar un contexto para las modificaciones que se introducen en \ref{sec:simulacion-fenomenos} y los problemas observados en \ref{sec:problemas-observados}.


\subsubsection{Envío de ACK}

Cuando la conexión está en estado \established y se recibe un paquete con payload, ocurre alguna de las siguientes situaciones:

\begin{itemize}
  \item Se ignora el payload, por ejemplo en caso que la ventana de recepción es nula o que los bytes del payload están fuera de la ventana de recepción.

  \item Se almacena total o parcialmente el payload recibido en el buffer de recepción y se reduce la ventana de recepción en tantos bytes como los que se hayan almacenado en ese buffer.
\end{itemize}

A su vez, en el caso que se almacenan bytes en el buffer de recepción, la implementación notifica al interlocutor enviando un paquete cuyo campo \ack tiene el valor del número de secuencia del próximo byte que se espera recibir y cuyo campo \window indica el nuevo tamaño de la ventana de recepción. Esto puede ocurrir de dos maneras distintas:

\begin{itemize}
  \item Cuando hay datos en el buffer de emisión, se aprovecha el envío del próximo paquete con datos al interlocutor (caso conocido como \emph{piggybacking}.)

  \item Cuando el buffer de emisión está vacío, se envía un paquete sin payload con el único propósito que el interlocutor reciba los valores de los campos \ack y \window.
\end{itemize}

La acción de enviar un paquete sin payload con el único propósito de comunicar la recepción de un paquete proveniente del interlocutor se refiere de ahora en más como \emph{envío de ACK}.


\subsubsection{Envío de paquete de actualización de ventana}
\label{sec:envio-paquete-ventana}

Cuando se leen datos del buffer de recepción, se libera el espacio del buffer previamente ocupado por los datos leídos y se aumenta la ventana de recepción proporcionalmente. Luego se envía al interlocutor un paquete sin payload con el único propósito de comunicar el nuevo valor de la ventana de recepción (campo \window.)

Esto ocurre independientemente de si se tienen datos en el buffer de salida o no.


\subsection{Simulación de los fenómenos estudiados}
\label{sec:simulacion-fenomenos}

Se modificó la implementación del protocolo para simular cada fenómeno de la siguiente manera:

\begin{description}
  \item[Latencia:] Se introduce un retraso en el envío de ACKs. Se suspende la ejecución durante cierta cantidad de tiempo inmediatamente antes del envío de los mismos.

  \item[Pérdida de paquetes:] Se descarta el envío de ACKs al azar, con cierta probabilidad.
\end{description}

Tanto el tiempo de retraso como la probabilidad de pérdida se proveen como parámetros al construir un socket PTC.


\subsection{Problemas observados}
\label{sec:problemas-observados}

Luego de introducir los cambios explicados en \ref{sec:simulacion-fenomenos}, se observaron algunos problemas durante la fase de experimentación que interferían con la correcta ejecución de los experimentos. Estos problemas en algunos casos fueron solucionados y en otros evitados implementando \emph{workarounds}. Explicamos los mismos a continuación.


\subsubsection{Error \texttt{[Errno 90] Message too long} causado por error de cálculo en bloque de control}

Se producía dicho error al intentar enviar un paquete con tamaño mayor a la MTU de Ethernet por medio del socket raw en la clase \texttt{Soquete}.

Este error ocurría porque, bajo ciertas circunstancias, el método \texttt{PTCControlBlock\#usable\_window\_size()} devolvía un valor negativo pequeño.

El valor devuelto por el método anterior representa la cantidad efectiva de bytes del buffer de emisión que se pueden enviar sin inundar al interlocutor. Dicho valor se emplea en el método \\ \texttt{PTCControlBlock\#extract\_from\_out\_buffer(size)} para extraer esa cantidad de bytes del buffer de emisión invocando al método \texttt{DataBuffer\#get(size)}, que recibe dicho valor como parámetro. Luego, los bytes extraídos del buffer se empaquetan como payload de un paquete PTC, y éste se envía al interlocutor. 

En circunstancias normales, el método \texttt{DataBuffer\#get(size)} extrae los primeros \texttt{size} bytes de una lista que contiene los bytes del buffer (tipo de dato \texttt{list} en Python.)

En el caso en el que ocurría el error, dicho método intentaba extraer una cantidad negativa de bytes del buffer, que debido a la semántica de las operaciones con índices en listas en Python, se interpretaba como la extracción de todos los bytes desde la posición 0 hasta \texttt{size} posiciones antes del final de la lista. Además, el buffer almacenaba cientos de miles de bytes al momento de producirse el error. Lo que ocurría finalmente es que se extraía una cantidad de bytes igual al número de bytes en el buffer menos el valor absoluto de \texttt{size}, que como este último era un valor pequeño, resultaba en el orden de los cientos de miles. Cuando luego se intentaba enviar un paquete PTC con un payload de tal tamaño, se producía el error \texttt{[Errno 90] Message too long}, pues la trama Ethernet resultante era de un tamaño muy superior a la MTU de Ethernet, que es de 1500 bytes.

No se encontró la causa de este problema. Sin embargo esta situación dejó de ocurrir luego de parchear el método \texttt{PTCControlBlock\#usable\_window\_size()} de forma que devolviera 0 en caso que el cálculo de ventana efectiva resultara negativo.


\subsubsection{Error \texttt{[Errno 90] Message too long} causado por \emph{maximum segment size} demasiado grande}
\label{sec:error-mss}

Este error ocurría porque por algún motivo cuya causa no se logró identificar, el método \\ \texttt{PTCControlBlock\#usable\_window\_size()} devolvía un tamaño efectivo de ventana de emisión muy grande, que producía que se enviaran paquetes PTC de tamaño superior al de la MTU de Ethernet.

Se logró evitar este problema modificando el valor de la constante \texttt{MSS}, que indica el tamaño máximo de segmento, de forma tal que los paquetes creados nunca superen el tamaño de la MTU.


\subsubsection{Crecimiento del tamaño de la ventana de recepción a medida que pasa el tiempo}

Ocurría que, durante la transmisión de un archivo entre dos hosts por medio de sockets PTC, la ventana de recepción del host que recibía el archivo eventualmente superaba su tamaño inicial y se incrementaba a medida que pasaba el tiempo. No se logró identificar la causa de este problema, pero se sospecha que se trata de un bug de sincronismo entre threads.

Se evitó la ocurrencia de este fenómeno modificando el bloque de control de forma de memorizar el tamaño inicial de la ventana de recepción, y utilizarlo como valor máximo en el método \\ \texttt{PTCControlBlock\#from\_in\_buffer(size)}, que extrae bytes del buffer de recepción e incrementa la ventana de recepción en función de la cantidad de bytes extraídos.

Luego de efectuar este parche, se conjeturó que el error explicado en \ref{sec:error-mss} se producía porque el interlocutor eventualmente recibía un tamaño de ventana de recepción que producía paquetes que superaban la MTU de Ethernet.


\subsubsection{\emph{Deadlock} por recepción de paquete con ventana nula}

En algunas ocasiones, sucedía durante la transmisión de un archivo entre dos hosts por medio de sockets PTC que el host emisor recibía un paquete con ventana nula, y luego no recibía más paquetes del host receptor. Esto producía que el host emisor quedara perpetuamente a la espera de un paquete con ventana no-nula proveniente del receptor para poder continuar la comunicación, y esto nunca ocurría.

La siguiente es una sucesión de eventos típicas que producía este problema:

\begin{enumerate}
  \item El host receptor tiene ventana de recepción máxima.
  
  \item El host emisor envía un paquete con payload de tamaño igual a la ventana de recepción del host receptor.

  \item El thread de recepción de paquetes del host receptor atiende el paquete recibido, lo almacena en el buffer de recepción y agota la ventana de recepción.

  \item Dicho thread queda suspendido antes de enviar el ACK correspondiente con ventana nula, simulando el fenómeno de latencia.

  \item El thread principal del host receptor, mientras tanto, extrae la totalidad de los bytes en el buffer de recepción, incrementa la ventana de recepción a su valor máximo y envía un paquete de actualización de ventana al host emisor, con valor de ventana máximo.

  \item Inmediatamente después, el thread de recepción de paquetes del host receptor reanuda su ejecución y envía el ACK con ventana nula.

  \item El host emisor recibe un paquete con ventana máxima seguido de otro con ventana nula. Este último anula la ventana de emisión del host emisor. Sin embargo en este instante el nodo receptor tiene ventana de recepción máxima.

  \item El host emisor queda a la espera de un paquete con ventana no-nula proveniente del receptor, pero éste sólo envía paquetes ACK o de actualización de ventana como respuesta a paquetes recibidos del host emisor.
\end{enumerate}

Como se puede observar, el último evento genera una situación de deadlock, en la que ambos hosts quedan perpetuamente a la espera de un paquete de su contraparte antes de volver a enviar nuevos paquetes.

Se solucionó este problema implementando un nuevo thread que envía periódicamente paquetes de actualización de ventana, denominados paquetes \emph{keepalive}, que informan al nodo emisor el valor correcto de ventana de recepción del nodo receptor, lo cual reanuda la transmisión del archivo.


\subsection{Experimentos realizados}
\label{sec:experimentos}

Los experimentos descritos a continuación fueron realizados con una versión del la implementación de PTC que incorpora tanto las modificaciones para simular los fenómenos estudiados, explicadas en \ref{sec:simulacion-fenomenos}, como las soluciones a los problemas descritos en \ref{sec:problemas-observados}.

Se realizó un conjunto de experimentos transmitiendo archivos de distintos tamaños desde un host \emph{cliente} hacia un host \emph{servidor} utilizando sockets PTC. Los experimentos fueron realizados en el contexto de una red LAN donde el cliente se conectaba a un router Linksys WRT54GL de forma inalámbrica (norma 802.11b), mientras que el servidor se conectaba al mismo router de forma cableada (norma 802.3, velocidad 100 Mbits/s.)

En los distintos experimentos se hicieron variar el tiempo de retraso de envíos de ACKs y la probabilidad de pérdida de los mismos en el host servidor, mientras que en el host cliente no se introdujeron retrasos ni pérdidas.

En todos los experimentos las mediciones de tiempo total de transmisión fueron realizadas usando la función \texttt{time} del módulo \texttt{time} de Python. Estas mediciones se realizan tomando el tiempo inmediatamente antes de comenzar a transmitir y luego justo después de cerrar el socket. El siguiente extracto de código ejemplifica lo explicado:

\begin{verbatim}
    t0 = time.time()          # Tiempo inicial
    sock.send(datos)
    sock.shutdown(SHUT_WR)
    t1 = time.time()          # Tiempo final
    t = t1 - t0               # Tiempo total de transferencia
\end{verbatim}

Los experimentos realizados se dividen en dos subconjuntos explicados a continuación.


\subsubsection{Experimento 1: Transmisión de archivos de distintos tamaños, sin retraso ni pérdida de ACKs}

Se transmitieron archivos de tamaños siguiendo la secuencia 50 KB, 100 KB, 150KB, 200 KB, \ldots, 950 KB, 1 MB. Se enviaron 10 archivos de cada tamaño, y para cada transmisión se registró el tiempo total de transmisión y la cantidad de retransmisiones de paquetes.


\subsubsection{Experimento 2: Transmisión de archivos de tamaño fijo, para distintos tiempos de retraso de envío y probabilidades de pérdida de ACKs}

Se transmitieron 10 archivos de 50 KB para cada par de tiempo de retraso y probabilidad de pérdida de ACKs según las secuencias a continuación:

\begin{itemize}
  \item Tiempos de retraso de envío de ACKs:  0 ms, 5 ms, 10 ms, 15 ms, \ldots, 95 ms, 100 ms.
  \item Probabilidades de pérdida de ACKs: 0, 0.05, 0.1, 0.15, \ldots, 0.45, 0.5.
\end{itemize}

Para cada transmisión se registró el tiempo total de transmisión y la cantidad de retransmisiones de paquetes.


%%%%%%%%%%%%%%%%%%%%%%%%%%%%%%%%%%%%%%%%%%%%%%%%%%%%%%%%%%%%%%%%%%%%%%%%%%%%%%%
%% Resultados                                                                %%
%%%%%%%%%%%%%%%%%%%%%%%%%%%%%%%%%%%%%%%%%%%%%%%%%%%%%%%%%%%%%%%%%%%%%%%%%%%%%%%


\section{Resultados}
\label{sec:resultados}

A continuación se presentan los resultados de los experimentos descritos en \ref{sec:experimentos}.


\subsection{Experimento 1: Transmisión de archivos de distintos tamaños, sin retraso ni pérdida de ACKs}
\label{sec:experimento-distintos-tamanos-de-archivo}

\grafico{time_vs_size}
        {Tiempo de transferencia en función del tamaño de archivo}
        {plot:time_vs_size}

\grafico{retransmissions_vs_size}
        {Retransmisiones en función del tamaño de archivo}
        {plot:retransmissions_vs_size}

Los picos que se observan en la figura \ref{plot:retransmissions_vs_size} se deben a la presencia de outliers cuya causa no fue identificada. Ver discusión en la sección \ref{sec:discusion-retransmisiones-vs-tam}.

Para visualizar mejor estos resultados, se filtraron todas las transmisiones con cantidad de retransmisiones superior a 10, que representan aproximadamente el 3\% del total de las mediciones. El subconjunto obtenido se ilustra en la figura \ref{plot:retransmissions_vs_size_wo_outliers}.

\grafico{retransmissions_vs_size_wo_outliers}
        {Retransmisiones en función del tamaño de archivo, excluyendo outliers}
        {plot:retransmissions_vs_size_wo_outliers}


\subsection{Experimento 2: Transmisión de archivos de tamaño fijo, para distintos tiempos de retraso de envío y probabilidades de pérdida de ACKs}

\grafico{time_vs_delay_and_loss_probability}
        {Tiempo de transferencia en función del retraso en envío de ACKs para distintas probabilidades de pérdida de ACKs}
        {plot:time_vs_delay_and_loss_probability}

\grafico{time_vs_delay_and_loss_probability_min_max}
        {Tiempo de transferencia en función del retraso en envío de ACKs para las probabilidades de pérdida de ACKs mínima y máxima}
        {plot:time_vs_delay_and_loss_probability_min_max}

\grafico{retransmissions_vs_delay_and_loss_probability}
        {Retransmisiones en función del retraso en envío de ACKs para distintas probabilidades de pérdida de ACKs}
        {plot:retransmissions_vs_delay_and_loss_probability}

Tal como en \ref{sec:experimento-distintos-tamanos-de-archivo}, no se conoce la causa de los picos observados en la figura \ref{plot:retransmissions_vs_delay_and_loss_probability}.

En la figura \ref{plot:retransmissions_vs_delay_and_loss_probability_wo_outliers} se excluyen las mediciones con cantidad de retransmisiones superior a 10, que representan aproximadamente el 2\% del total de las mediciones realizadas en este experimento.

\grafico{retransmissions_vs_delay_and_loss_probability_wo_outliers}
        {Retransmisiones en función del retraso en envío de ACKs para distintas probabilidades de pérdida de ACKs, excluyendo outliers}
        {plot:retransmissions_vs_delay_and_loss_probability_wo_outliers}


%%%%%%%%%%%%%%%%%%%%%%%%%%%%%%%%%%%%%%%%%%%%%%%%%%%%%%%%%%%%%%%%%%%%%%%%%%%%%%%
%% Discusión                      			                                     %%
%%%%%%%%%%%%%%%%%%%%%%%%%%%%%%%%%%%%%%%%%%%%%%%%%%%%%%%%%%%%%%%%%%%%%%%%%%%%%%%


\section{Discusión}
\label{sec:discusion}

Se comentan a continuación algunas observaciones sobre los resultados de los experimentos.


\subsection{Tiempo de transferencia en función del tamaño del archivo}

En la figura \ref{plot:time_vs_size} podemos observar cómo el tiempo de transferencia crece linealmente al incrementar el tamaño del archivo transmitido. Este comportamiento era el esperado, dado que para archivos cada vez más grandes, las transmisiones requieren proporcionalmente más envíos de paquetes para ser completadas.


\subsection{Retransmisiones en función del tamaño de archivo}
\label{sec:discusion-retransmisiones-vs-tam}

No se sabe con certeza la causa de los picos en la figura \ref{plot:retransmissions_vs_size} (tamaños de archivo 800, 900 y 950 KB.) Se sospecha que la causa es un deterioro en la calidad de la señal Wi-Fi entre el host cliente y el router\footnote{Topología de la red descrita en \ref{sec:experimentos}}. La transmisión de los archivos se realizó de forma secuencial, por tamaño de archivo en orden creciente. Entonces una posible explicación es que la señal haya comenzado a deteriorarse en algún punto durante la transmisión del archivo de 800 KB, y que se haya normalizado durante la transmisión del archivo de tamaño 950 KB.

Dichos picos fueron filtrados en la figura \ref{plot:retransmissions_vs_size_wo_outliers}. En ésta se observa que la cantidad de retransmisiones no parece verse afectada por el tamaño del archivo transmitido. Esto resulta sorprendente, ya que por la intuición que se tenía del funcionamiento del protocolo, se esperaba la cantidad de retransmisiones fuera proporcional al volumen de datos enviados.

Se desconoce la causa de la varianza en dicha figura (ver tamaños de archivo 200 y 300 KB), pero se sospecha que se debe a la pequeña cantidad de repeticiones realizadas durante los experimentos (10 repeticiones por cada tamaño de archivo.)


\subsection{Tiempo de transferencia en función del tiempo de retraso en envío de ACKs para distintas probabilidades de pérdida de ACKs}

En las figuras \ref{plot:time_vs_delay_and_loss_probability} y \ref{plot:time_vs_delay_and_loss_probability_min_max} se observan dos resultados interesantes:

\begin{itemize}
  \item El tiempo de transferencia crece linealmente en la medida que crece el tiempo de retraso de envío de ACKs.

  \item El tiempo de transferencia decrece en la medida que crece la probabilidad de pérdida de ACKs.
\end{itemize}

Ambos hechos contradicen la intuición que se tenía sobre el funcionamiento del protocolo antes de realizar los experimentos: sólo se retrasan y/o pierden los paquetes ACK. Los paquetes de actualización de ventana (ver \ref{sec:envio-paquete-ventana}) son enviados sin retrasos ni pérdidas. Estos paquetes también acarrean el valor actualizado del próximo número de secuencia esperado (campo \ack). Ahora bien, los paquetes de actualización de ventana se envían cada vez que un host extrae bytes de su buffer de entrada y libera espacio en el mismo. Pero como en estos experimentos el servidor está leyendo constantemente de su buffer de entrada, cuando el servidor recibe un paquete del cliente con datos, éste contesta inmediatamente con un paquete de actualización de ventana.

En conclusión, el servidor contesta con un paquete ACK y con un paquete de actualización de ventana cada vez que recibe un paquete con datos del cliente\footnote{Esto es lo que ocurre cuando no se está simulando retraso ni pérdida de ACKs.}. El paquete de actualización de ventana entonces actúa como backup para comunicar el próximo número de secuencia esperado, y el cliente siempre recibe ese número inmediatamente después que el servidor recibe datos del cliente, por más que se retrasen o pierdan los paquetes ACK.

Luego el tiempo de transmisión no debería verse afectado por el retraso o pérdida de ACKs, ya que el cliente nunca debería quedarse esperando la recepción de un ACK demorado para continuar con la transmisión, ni deberían ocurrir timeouts y retransmisiones por la pérdida de los mismos.

Esta intuición se explica con más detalle en la sección \ref{sec:discusion-intuicion}. 

No se logró hipotetizar una causa para el crecimiento del tiempo de transferencia respecto del crecimiento del tiempo de retraso en envío de ACKs. Se realizó un análisis informal del código de la implementación del protocolo, bajo la sospecha que la suspensión del thread de recepción de paquetes antes del envío de un ACK pudiera estar bloqueando el acceso a algún recurso utilizado por los otros threads hasta la reanudación del mismo. Esto explicaría el aumento del tiempo de transferencia a medida que aumenta el tiempo de retraso. Sin embargo este análisis no corroboró dicha sospecha.

En cuanto a la reducción del tiempo de transferencia respecto del crecimiento de la probabilidad de pérdida de ACKs, se sospecha que se debe a una reducción del uso de CPU tanto en el cliente como en el servidor\footnote{Recordar que el stack PTC está programado en lenguaje Python, que no se destaca por ser rápido. Además éste corre como aplicación en el espacio de usuario en lugar de formar parte del sistema operativo, con la penalidad de performance que eso implica.} como consecuencia de procesar menos paquetes. Además, se sospecha que el menor tráfico en la red resultante de transmitir menos paquetes contribuye en la reducción del tiempo total de transferencia.


\subsection{Retransmisiones en función del tiempo de retraso en envío de ACKs para distintas probabilidades de pérdida de ACKs}

De manera similar a \ref{sec:discusion-retransmisiones-vs-tam}, los picos observados en la figura \ref{plot:retransmissions_vs_delay_and_loss_probability} se atribuyen al deterioro momentáneo de la calidad de la señal Wi-Fi entre el host cliente y el router.

Dichos picos fueron filtrados en la figura \ref{plot:retransmissions_vs_delay_and_loss_probability_wo_outliers}. No se observa relación alguna entre cantidad de retransmisiones y tiempo de retraso en envío de ACKs, ni entre cantidad de retransmisiones y probabilidad de pérdida de ACKs.


\subsection{Intuición del funcionamiento del protocolo previa a la experimentación}
\label{sec:discusion-intuicion}

La siguiente sucesión de eventos describe un escenario típico que ocurre cuando un host recibe un paquete PTC con payload no vacío, asumiendo que no hay pérdida ni reordenamiento de paquetes.

\begin{enumerate}
  \item El host agrega los datos recibidos a su buffer de entrada.
  
  \item Si su buffer de salida está vacío, el host envía un paquete sin datos, tal que su campo \ack indica el número de secuencia del próximo byte que se espera recibir.
  
  \item Si su buffer de salida no está vacío, lo anterior no ocurre y simplemente el próximo paquete a emitir con datos del buffer de salida comunicará en su campo \ack dicho número de secuencia (piggybacking.)
  
  \item Eventualmente la aplicación lee del buffer de entrada, liberando espacio en el mismo. Cuando esto ocurre, el host envía un paquete sin datos (independiente de si hay datos en el buffer de salida o no) con el nuevo valor de ventana correspondiente, con el único propósito de actualizar la ventana de emisión del host del otro lado de la conexión.
\end{enumerate}

En el contexto de los experimentos realizados, el host servidor nunca envía datos al cliente, y lee constantemente del buffer de entrada hasta que se reciben la totalidad de los bytes esperados. En consecuencia, siempre ocurre el paso 2 y nunca ocurre el paso 3 (piggybacking), pues el buffer de salida siempre está vacío. Además, el paso 4 ocurre inmediatamente después del paso 2, porque la aplicación lee continuamente de su buffer de entrada, y los bytes del buffer de entrada son desalojados inmediatamente después de ser recibidos.

Es decir, inmediatamente después de recibir un paquete con datos, el servidor responde con dos paquetes sin datos:

\begin{itemize}
  \item El paquete ACK, cuyo campo \ack refleja el reconocimiento de los últimos datos recibidos y su campo ventana decrementado en función a la cantidad de bytes recibidos, reflejando la adición de esos bytes al buffer de entrada del servidor.

  \item El paquete de actualización de ventana, idéntico al ACK salvo por el campo ventana, que refleja el incremento e la ventana de recepción luego de extraer esos bytes del buffer de entrada.
\end{itemize}

Tiene sentido entonces, en el contexto de los experimentos realizados, pensar el paquete de actualización de ventana como un segundo paquete ACK que actúa como backup del primero.

Considerar, a modo de ejemplo, la siguiente captura hipotética durante alguno de los experimentos realizados sin retraso ni pérdida de ACKs, entre un host cliente con IP 192.168.0.1 y un host servidor con IP 10.0.0.1. Cada paquete enviado por el cliente agota la totalidad de la ventana de recepción del servidor, y el cliente queda a la espera de la recepción de un paquete del servidor con ventana no nula antes de poder enviar el siguiente paquete con datos.

\begin{Verbatim}[fontsize=\small]
1. [ SRC=192.168.0.1 | DST=10.0.0.1    | Flags=ACK | #SEQ=20000 | #ACK=100   | WND=9999 | LEN=1024 ]
2. [ SRC=10.0.0.1    | DST=192.168.0.1 | Flags=ACK | #SEQ=100   | #ACK=21024 | WND=0    | LEN=0    ]
3. [ SRC=10.0.0.1    | DST=192.168.0.1 | Flags=ACK | #SEQ=100   | #ACK=21024 | WND=1024 | LEN=0    ]
4. [ SRC=192.168.0.1 | DST=10.0.0.1    | Flags=ACK | #SEQ=21024 | #ACK=100   | WND=9999 | LEN=1024 ]
5. [ SRC=10.0.0.1    | DST=192.168.0.1 | Flags=ACK | #SEQ=100   | #ACK=22048 | WND=0    | LEN=0    ]
6. [ SRC=10.0.0.1    | DST=192.168.0.1 | Flags=ACK | #SEQ=100   | #ACK=22048 | WND=1024 | LEN=0    ]
\end{Verbatim}

Suponer ahora que durante el experimento anterior el servidor retrasa $n$ segundos el envío de los paquetes ACK (paquetes 2 y 5). La siguiente captura refleja este escenario.

\begin{Verbatim}[fontsize=\small]
   [ SRC=192.168.0.1 | DST=10.0.0.1    | Flags=ACK | #SEQ=20000 | #ACK=100   | WND=9999 | LEN=1024 ]
   [ SRC=10.0.0.1    | DST=192.168.0.1 | Flags=ACK | #SEQ=100   | #ACK=21024 | WND=1024 | LEN=0    ]
   [ SRC=192.168.0.1 | DST=10.0.0.1    | Flags=ACK | #SEQ=21024 | #ACK=100   | WND=9999 | LEN=1024 ]
   [ SRC=10.0.0.1    | DST=192.168.0.1 | Flags=ACK | #SEQ=100   | #ACK=22048 | WND=1024 | LEN=0    ]
\end{Verbatim}

$\vdots$ 

($n$ segundos después)

\begin{Verbatim}[fontsize=\small]
   [ SRC=10.0.0.1    | DST=192.168.0.1 | Flags=ACK | #SEQ=100   | #ACK=21024 | WND=0    | LEN=0    ]
   [ SRC=10.0.0.1    | DST=192.168.0.1 | Flags=ACK | #SEQ=100   | #ACK=22048 | WND=0    | LEN=0    ]
\end{Verbatim}

Si bien los paquetes ACK se ven demorados, los paquetes de actualización de ventana llegan instantáneamente. Estos paquetes no sólo informan que se recibieron correctamente los datos transmitidos (campo \ack) sino que además anuncian una ventana de recepción no nula. \emph{Por lo tanto el cliente puede continuar transmitiendo normalmente y sin demoras}.

En el caso que se estuviera simulando pérdida de paquetes ACK, dichos paquetes podrían pensarse como paquetes que son demorados infinitamente, por lo que el análisis de la situación es idéntico al anterior.

En conclusión, parece razonable intuír que ni el retraso ni la pérdida simulada de ACKs deberían afectar el tiempo de transferencia de un archivo, ya que por más que se retrasen o pierdan ACKs, los paquetes de actualización de ventana actúan como backup de éstos, y el cliente puede seguir transmitiendo sin problemas.

Sin embargo, de acuerdo a lo discutido en las secciones anteriores, ésta intuición no verifica los resultados obtenidos luego de la experimentación.


%%%%%%%%%%%%%%%%%%%%%%%%%%%%%%%%%%%%%%%%%%%%%%%%%%%%%%%%%%%%%%%%%%%%%%%%%%%%%%%
%% Conclusión                                                                %%
%%%%%%%%%%%%%%%%%%%%%%%%%%%%%%%%%%%%%%%%%%%%%%%%%%%%%%%%%%%%%%%%%%%%%%%%%%%%%%%


\section{Conclusión}

Las conclusiones a continuación refieren a la implementación del protocolo PTC luego de efectuar las modificaciones explicadas en \ref{sec:simulacion-fenomenos} y \ref{sec:problemas-observados}:

\begin{itemize}
  \item El tiempo de transferencia de un archivo entre dos hosts utilizando sockets PTC crece linealmente respecto del tamaño del archivo.

  \item El tiempo de transferencia de un archivo entre dos hosts utilizando sockets PTC crece linealmente respecto del retraso en el envío de ACKs.

  \item No se observa correlación entre la cantidad de retransmisiones y el tamaño del archivo transmitido.

  \item El tiempo de transferencia del archivo decrece en la medida que crece la probabilidad de pérdida de ACKs provenientes del host destino.

  \item El tiempo de transferencia del archivo crece en la medida que crece el tiempo de retraso en el envío de ACKs.
\end{itemize}

Notar que las modificaciones ad-hoc realizadas sobre la implementación del protocolo para solucionar los problemas descritos en \ref{sec:problemas-observados} alteran el funcionamiento del protocolo. Por este motivo, las conclusiones anteriores no necesariamente aplican a posibles versiones futuras de dicha implementación que solucionen las causas de los problemas observados en lugar de meramente evitar los síntomas de los mismos.


%%%%%%%%%%%%%%%%%%%%%%%%%%%%%%%%%%%%%%%%%%%%%%%%%%%%%%%%%%%%%%%%%%%%%%%%%%%%%%%
%% Trabajo futuro                                                            %%
%%%%%%%%%%%%%%%%%%%%%%%%%%%%%%%%%%%%%%%%%%%%%%%%%%%%%%%%%%%%%%%%%%%%%%%%%%%%%%%


\section{Trabajo futuro}

Se propone realizar un análisis de la interacción entre los threads de la implementación del protocolo PTC para descartar la existencia de los problemas de sincronismo que pudieran causar algunos de los problemas mencionados en \ref{sec:problemas-observados}. Luego de esto, se propone crear una nueva iteración de la implementación del protocolo PTC solucionando correctamente los problemas mencionados.

Con esta nueva iteración, se propone repetir los experimentos realizados en este trabajo y estudiar las diferencias encontradas, con la esperanza de poder entender mejor los fenómenos a los que no se les encontró una explicación.

Además, se propone extender la simulación de latencia y pérdida de paquetes para incluir a los paquetes de actualización de ventana, de manera de simular más fielmente los fenómenos en redes de área local cuyo impacto en el protocolo PTC se estudia en este trabajo.

Finalmente se propone estudiar los efectos de simular dichos fenómenos en ambos extremos de la conexión.


\end{document}